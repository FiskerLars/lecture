\svnInfo $Id: methodology.tex 22975 2012-01-14 19:43:38Z kohlhase $
\svnKeyword $HeadURL: https://svn.kwarc.info/repos/kwarc/doc/macros/forCTAN/proposal/eu/examples/strep/methodology.tex $
\section{Scientific/Technical Methodology and Work Plan}\label{sec:methodology}
\begin{todo}{from the proposal template}
  A detailed work plan should be presented, broken down into work packages\footnote{A work
    package is a major sub-division of the proposed project with a verifiable end-point –
    normally a deliverable or an important milestone in the overall project.} (WPs) which
  should follow the logical phases of the implementation of the project, and include
  consortium management and assessment of progress and results. (Note that your overall
  approach to management will be described later, in Section 2).

Notes: The number of work packages used must be appropriate to the complexity of the work
and the overall value of the proposed project. The planning should be sufficiently
detailed to justify the proposed effort and allow progress monitoring by the Commission.

Any significant risks should be identified, and contingency plans described
\end{todo}
\newpage\svnInfo $Id: workplan.tex 22929 2012-01-08 09:25:45Z kohlhase $
\svnKeyword $HeadURL: https://svn.kwarc.info/repos/kwarc/doc/macros/forCTAN/proposal/eu/examples/fetopenstrep/workplan.tex $

\begin{todo}{from the proposal template}
\begin{enumerate}
\item Describe the overall strategy of the work plan\ednote{Maximum length – one page}
\item Show the timing of the different WPs and their components (Gantt chart or similar).
\end{enumerate}
\end{todo}
\begin{figure}
  \caption{Work package dependencies}
  \label{fig:wp-deps}
\end{figure}

\ganttchart[draft,xscale=.45] 

%%% Local Variables: 
%%% mode: LaTeX
%%% TeX-master: "propB"
%%% End: 

% LocalWords:  workplan.tex ednote wp-deps ganttchart xscale


\newpage
\subsection{Work Package List}\label{sec:wplist}

\begin{todo}{from the proposal template}
Please indicate one activity per work package:
RTD = Research and technological development; DEM = Demonstration; MGT = Management of the consortium
\end{todo}

%\makeatletter\wp@total@RM{management}\makeatother
\wpfigstyle{\footnotesize}
\wpfig[pages,type,start,end]

\newpage\svnInfo $Id: deliverables.tex 24014 2013-01-01 15:12:26Z kohlhase $
\svnKeyword $HeadURL: https://svn.kwarc.info/repos/kwarc/doc/macros/forCTAN/proposal/eu/examples/strep/deliverables.tex $
\subsection{List of Deliverables}\label{sec:deliverables}

\begin{todo}{from the proposal template}
\begin{compactenum}
\item Deliverable numbers in order of delivery dates. Please use the numbering convention <WP number>.<number of deliverable within
that WP>. For example, deliverable 4.2 would be the second deliverable from work package 4.
\item Please indicate the nature of the deliverable using one of the following codes:
R = Report, P = Prototype, D = Demonstrator, O = Other
\item Please indicate the dissemination level using one of the following codes:
PU = Public
PP = Restricted to other programme participants (including the Commission Services).
RE = Restricted to a group specified by the consortium (including the Commission Services).
CO = Confidential, only for members of the consortium (including the Commission Services).
\end{compactenum}
\end{todo}
We will now give an overview over the deliverables and milestones of the work
packages. Note that the times of deliverables after month 24 are estimates and may change
as the work packages progress.

In the table below, {\emph{integrating work deliverables}} (see top of
section~\ref{sec:wplist}) are printed in boldface to mark them. They integrate
contributions from multiple work packages. \ednote{CL: the rest of this paragraph does not
  comply with the EU guide for applicants, needs to be rewritten}These can have the
dissemination level ``partial'', which indicates that it contains parts of level
``project'' that are to be disseminated to the project and evaluators only. In such
reports, two versions are prepared, and disseminated accordingly.

{\footnotesize\inputdelivs{8cm}}


%%% Local Variables: 
%%% mode: latex
%%% TeX-master: "propB"
%%% End: 

\newpage\svnInfo $Id: milestones.tex 24014 2013-01-01 15:12:26Z kohlhase $
\svnKeyword $HeadURL: https://svn.kwarc.info/repos/kwarc/doc/macros/forCTAN/proposal/eu/examples/strep/milestones.tex $
\subsection{List of Milestones}\label{sec:milestones}

\begin{todo}{from the proposal template}
  Milestones are control points where decisions are needed with regard to the next stage
  of the project. For example, a milestone may occur when a major result has been
  achieved, if its successful attainment is a requirement for the next phase of
  work. Another example would be a point when the consortium must decide which of several
  technologies to adopt for further development.

  Means of verification: Show how you will confirm that the milestone has been
  attained. Refer to indicators if appropriate. For examples: a laboratory prototype
  completed and running flawlessly, software released and validated by a user group, field
  survey complete and data quality validated.
\end{todo}


The work in the {\pn} project is structured by seven milestones, which coincide with the
project meetings in summer and fall.  Since the meetings are the main face-to-face
interaction points in the project, it is suitable to schedule the milestones for these
events, where they can be discussed in detail. We envision that this setup will give the
project the vital coherence in spite of the broad mix of disciplinary backgrounds of the
participants.\ednote{maybe automate the milestones}

\begin{milestones}
  \milestone[id=kickoff,verif=Inspection,month=1]
    {Initial Infrastructure}
    {Set up the organizational infrastructure, in particular: Web Presence, project TRAC,\ldots}
  \milestone[id=consensus,verif=Inspection,month=24]{Consensus} {Reach Consensus on the
    way the project goes}
  \milestone[id=exploitation,verif=Inspection,month=36]{Exploitation}{The exploitation
    plan should be clear so that we can start on this in the last year.}
  \milestone[id=final,verif=Inspection,month=48]{Final Results}{all is done}
\end{milestones}

%%% Local Variables: 
%%% mode: latex
%%% TeX-master: "propB"
%%% End: 

% LocalWords:  pn ednote verif ldots


\subsection{Work Package Descriptions}\label{sec:workpackages}
\begin{workplan}
\svnInfo $Id: wp-management.tex 24014 2013-01-01 15:12:26Z kohlhase $
\svnKeyword $HeadURL: https://svn.kwarc.info/repos/kwarc/doc/macros/forCTAN/proposal/eu/examples/strep/wp-management.tex $
\begin{workpackage}[id=management,type=MGT,wphases=0-24!.2,
  title=Project Management,short=Management,
  jacuRM=2,barRM=2,efoRM=2,bazRM=2]
We can state the state of the art and similar things before the summary in the boxes
here. 
\wpheadertable
\begin{wpobjectives}
  \begin{itemize}
    \item To perform the administrative, scientific/technical, and financial
      management of the project
    \item To co-ordinate the contacts with the EU
    \item To control quality and timing of project results and to resolve conflicts
    \item To set up inter-project communication rules and mechanisms
  \end{itemize}
\end{wpobjectives}

\begin{wpdescription}
  Based on the Consortium Agreement, i.e. the contract with the European Commission, and
  based on the financial and administrative data agreed, the project manager will carry
  out the overall project management, including administrative management.  A project
  quality handbook will be defined, and a {\pn} help-desk for answering questions about
  the format (first project-internal, and after month 12 public) will be established. The
  project management will\ldots we can even reference deliverables:
  \delivref{management}{report2} and even the variant with a title:
  \delivtref{management}{report2}
\end{wpdescription}

\begin{wpdelivs}
  \begin{wpdeliv}[due=1,id=mailing,nature=O,dissem=PP,miles=kickoff]
    {Project-internal mailing lists}
  \end{wpdeliv}
  \begin{wpdeliv}[due=3,id=handbook,nature=R,dissem=PU,miles=consensus]
    {Project management handbook}
  \end{wpdeliv}
\begin{wpdeliv}[due={6,12,18,24,30,36,42,48},id=report2,nature=R,dissem=public,miles={consensus,final}]
  {Periodic activity report} 
  Partly compiled from activity reports of the work package
  coordinators; to be approved by the work package coordinators before delivery to the
  Commission.  Financial reporting is mainly done in months 18 and 36.\Ednote{how about
    these numbers?}
  \end{wpdeliv}
 \begin{wpdeliv}[due=6,id=helpdesk,dissem=PU,nature=O,miles=kickoff]
    {{\pn} Helpdesk}
  \end{wpdeliv}
  \begin{wpdeliv}[due=36,id=report6,nature=R,dissem=PU,miles=final]
    {Final plan for using and disseminating the knowledge}
  \end{wpdeliv}
  \begin{wpdeliv}[due=48,id=report7,nature=R,dissem=PU,miles=final]
    {Final management report}
  \end{wpdeliv}
\end{wpdelivs}
\end{workpackage}

%%% Local Variables: 
%%% mode: LaTeX
%%% TeX-master: "propB"
%%% End: 

% LocalWords:  wp-management.tex workpackage efoRM bazRM wpheadertable pn ldots
% LocalWords:  wpobjectives wpdescription delivref delivtref wpdelivs wpdeliv
% LocalWords:  dissem Ednote pdataRef deliv mansubsusintReport wphases
\newpage
\svnInfo $Id: wp-dissem.tex 24014 2013-01-01 15:12:26Z kohlhase $
\svnKeyword $HeadURL: https://svn.kwarc.info/repos/kwarc/doc/macros/forCTAN/proposal/eu/examples/strep/wp-dissem.tex $
\begin{workpackage}%
[id=dissem,type=RTD,lead=efo,
 wphases=10-24!1,
 title=Dissemination and Exploitation,short=Dissemination,
 efoRM=8,jacuRM=2,barRM=2,bazRM=2]
We can state the state of the art and similar things before the summary in the boxes
here. 
\wpheadertable

\begin{wpobjectives}
  Much of the activity of a project involves small groups of nodes in joint work. This
  work package is set up to ensure their best wide-scale integration, communication, and
  synergetic presentation of the results. Clearly identified means of dissemination of
  work-in-progress as well as final results will serve the effectiveness of work within
  the project and steadily improve the visibility and usage of the emerging semantic
  services.
\end{wpobjectives}

\begin{wpdescription}
  The work package members set up events for dissemination of the research and
  work-in-progress results for researchers (workshops and summer schools), and for
  industry (trade fairs). An in-depth evaluation will be undertaken of the response of
  test-users.

  Within two months of the start of the project, a project website will go live. This
  website will have two areas: a members' area and a public area.\ldots
\end{wpdescription}

\begin{wpdelivs}
  \begin{wpdeliv}[due=2,id=website,nature=O,dissem=PU,miles=kickoff]
     {Set-up of the Project web server}
   \end{wpdeliv}
   \begin{wpdeliv}[due=8,id=ws1proc,nature=R,dissem=PU,miles={kickoff}]
     {Proceedings of the first {\pn} Summer School.}
   \end{wpdeliv}
   \begin{wpdeliv}[due=9,id=dissem,nature=R,dissem=PP]
     {Dissemination Plan}
   \end{wpdeliv}
   \begin{wpdeliv}[due=9,id=exploitplan,nature=R,dissem=PP,miles=exploitation]
     {Scientific and Commercial Exploitation Plan}
   \end{wpdeliv}
   \begin{wpdeliv}[due=20,id=ws2proc,nature=R,dissem=PU,miles={exploitation}]
     {Proceedings of the second {\pn} Summer School.}
   \end{wpdeliv}
   \begin{wpdeliv}[due=32,id=ss1proc,nature=R,dissem=PU,miles={exploitation}]
     {Proceedings of the third {\pn} Summer School.}
   \end{wpdeliv}
   \begin{wpdeliv}[due=44,id=ws3proc,nature=R,dissem=PU,miles=exploitation]
     {Proceedings of the fourth {\pn} Summer School.}
   \end{wpdeliv}
 \end{wpdelivs}
\end{workpackage}

%%% Local Variables: 
%%% mode: LaTeX
%%% TeX-master: "propB"
%%% End: 

% LocalWords:  wp-dissem.tex workpackage dissem efo fromto bazRM wpheadertable
% LocalWords:  wpobjectives wpdescription ldots wpdelivs wpdeliv ws1proc pn
% LocalWords:  exploitplan ws2proc ss1proc ws3proc pdataRef deliv
% LocalWords:  mansubsusintReport
\newpage
\svnInfo $Id: wp-class.tex 24014 2013-01-01 15:12:26Z kohlhase $
\svnKeyword $HeadURL: https://svn.kwarc.info/repos/kwarc/doc/macros/forCTAN/proposal/eu/examples/strep/wp-class.tex $
\begin{workpackage}[id=class,type=RTD,lead=jacu,
                    wphases=3-9!1,
                    title=A {\LaTeX} class for EU Proposals,short=Class,
                    jacuRM=12,barRM=12]
We can state the state of the art and similar things before the summary in the boxes
here. 
\wpheadertable
\begin{wpobjectives}
\LaTeX is the best document markup language, it can even be used for literate
programming~\cite{DK:LP,Lamport:ladps94,Knuth:ttb84}

  To develop a {\LaTeX} class for marking up EU Proposals
\end{wpobjectives}

\begin{wpdescription}
  We will follow strict software design principles, first comes a requirements analys,
  then \ldots
\end{wpdescription}

\begin{wpdelivs}
  \begin{wpdeliv}[due=6,id=req,nature=R,dissem=PP,miles=kickoff]
     {Requirements analysis}
   \end{wpdeliv}
   \begin{wpdeliv}[due=12,id=spec,nature=R,dissem=PU,miles=consensus]
     {{\pn} Specification }
   \end{wpdeliv}
   \begin{wpdeliv}[due=18,id=demonstrator,nature=P,dissem=PU,miles={consensus,final}]
     {First demonstrator ({\tt{article.cls}} really)}
   \end{wpdeliv}
   \begin{wpdeliv}[due=24,id=proto,nature=P,dissem=PU,miles=final]
     {First prototype}
   \end{wpdeliv}
    \begin{wpdeliv}[due=36,id=release,nature=P,dissem=PU,miles=final]
      {Final {\LaTeX} class, ready for release}
    \end{wpdeliv}
  \end{wpdelivs}
Furthermore, this work package contributes to {\pdataRef{deliv}{managementreport2}{label}} and
{\pdataRef{deliv}{managementreport7}{label}}.
\end{workpackage}

%%% Local Variables: 
%%% mode: LaTeX
%%% TeX-master: "propB"
%%% End: 
\newpage
\svnInfo $Id: wp-temple.tex 24014 2013-01-01 15:12:26Z kohlhase $
\svnKeyword $HeadURL: https://svn.kwarc.info/repos/kwarc/doc/macros/forCTAN/proposal/eu/examples/strep/wp-temple.tex $
\begin{workpackage}[id=temple,type=DEM,lead=bar,
  wphases=6-12!1,
  title={\pn} Proposal Template,short=Template,barRM=6,bazRM=6]
We can state the state of the art and similar things before the summary in the boxes
here. 
\wpheadertable

\begin{wpobjectives}
  To develop a template file for {\pn} proposals
\end{wpobjectives}

\begin{wpdescription}
  We abstract an example from existing proposals
\end{wpdescription}

\begin{wpdelivs}
  \begin{wpdeliv}[due=6,id=req,nature=R,dissem=PP,miles=kickoff]
    {Requirements analysis}
  \end{wpdeliv}
  \begin{wpdeliv}[due=12,id=spec,nature=R,dissem=PU,miles=consensus]
    {{\pn} Specification }
  \end{wpdeliv}
  \begin{wpdeliv}[due=18,id=demonstrator,nature=D,dissem=PU,miles={consensus,final}]
    {First demonstrator ({\tt{article.cls}} really)}
  \end{wpdeliv}
  \begin{wpdeliv}[due=24,id=proto,nature=P,dissem=PU,miles=final]
    {First prototype}
  \end{wpdeliv}
  \begin{wpdeliv}[due=36,id=release,nature=P,dissem=PU,miles=final]
    {Final Template, ready for release}
  \end{wpdeliv}
\end{wpdelivs}
Furthermore, this work package contributes to {\pdataRef{deliv}{managementreport2}{label}} and
{\pdataRef{deliv}{managementreport7}{label}}.
\end{workpackage}

%%% Local Variables: 
%%% mode: LaTeX
%%% TeX-master: "propB"
%%% End: 

% LocalWords:  wp-temple.tex workpackage fromto pn bazRM wpheadertable wpdelivs
% LocalWords:  wpobjectives wpdescription wpdeliv req dissem tt article.cls
% LocalWords:  pdataRef deliv systemsintReport
\newpage
\end{workplan}
\newpage\svnInfo $Id: risks.tex 21553 2011-04-30 05:50:58Z kohlhase $
\svnKeyword $HeadURL: https://svn.kwarc.info/repos/kwarc/doc/macros/forCTAN/proposal/eu/examples/strep/risks.tex $
\subsection{Significant Risks and Associated Contingency Plans}\label{sec:risks}
\begin{todo}{from the proposal template}
  Describe any significant risks, and associated contingency plans
\end{todo}
\begin{oldpart}{need to integrate this somewhere. CL: I will check other proposals to see how they did it; the Guide does not really prescribe anything.}
\paragraph{Global Risk Management}
The crucial problem of \pn (and similar endeavors that offer a new basis for communication
and interaction) is that of community uptake: Unless we can convince scientists and
knowledge workers industry to use the new tools and interactions, we will
never be able to assemble the large repositories of flexiformal mathematical knowledge we
envision. We will consider uptake to be the main ongoing evaluation criterion for the network.
\end{oldpart}

%%% Local Variables: 
%%% mode: latex
%%% TeX-master: "propB"
%%% End: 



%%% Local Variables: 
%%% mode: latex
%%% TeX-master: "propB"
%%% End: 

% LocalWords:  workplan newpage wplist makeatletter makeatother wpfig
% LocalWords:  workpackages wp-dissem wp-class wp-temple
