%%%%%%%%%%%%%%%%%%%%%%%%%%%%%%%%%%%%%%%%%%%%%%%%%%%%%%%%%%%
% Datei f�r Definitionen und Commandos um Begriffe        %
% zu Vereinheitlichen                                     %
%                                                         %
% $Id: defs.tex,v 1.15 2003/04/15 14:25:23 tiga Exp $
%%%%%%%%%%%%%%%%%%%%%%%%%%%%%%%%%%%%%%%%%%%%%%%%%%%%%%%%%%%

\usepackage{enumerate}
\usepackage{amsmath,amsfonts,amssymb}
\usepackage{graphicx}
\usepackage{xcolor}
\usepackage{url}
%\usepackage{theorem}
%\usepackage{ulsy} % f. \blitza,...

% Theorem  %%%%%%%%%%%%%%%%%%%%%%%%%%%%%

%   \theoremstyle{break}
%    \newtheorem{Def}{Definition}
%    \newtheorem{Con}{Vermutung}
%    \newtheorem{Lem}{Lemma}
%    \newtheorem{Theorem}{Satz}
%   \newtheorem{Proof}{Beweis}

% \theorembodyfont{\itshape}
% \theoremheaderfont{\scshape}



%%%%%%%%%%%%%%%%%%%%%%%%%%%%%%%%%%%%%%%%

\newcommand{\cfbox}[2]{%
    \colorlet{currentcolor}{.}%
    {\color{#1}%
    \fbox{\color{currentcolor}#2}}%
}

% allgemein
%\newcommand{\FIXME}[1]{{\bf\color{red}{FIXME:} #1}}
\newcommand{\NOTE}[1]{\qquad{\bf \large $\bullet$NOTE: }\texttt{#1}}
\newcommand{\code}[1]{\verb+#1+}
\newcommand{\marginlabel}[1] {\mbox{}\marginpar{\raggedleft\hspace{0pt}#1}}

%dummy-picture
\newcommand{\dummypict}{\center{\framebox[7cm][c]{\bf {\large FIXME:} Dies ist ein Dummy\vspace{5cm}}}}



%some abbrevs
\newcommand{\wrt}[0]{{w.\,r.\,t.} }
\newcommand{\eg}[0]{{e.\,g.,} }
\newcommand{\etal}[0]{{~et~al.} }
\newcommand{\ie}[0]{{i.\,e.,} }



\newcommand{\doublesignature}[3]{%
  \parbox{\textwidth}{
    %\centering 
    #3, \today\\[2cm]
    \parbox{7cm}{
      \centering
      \rule{6cm}{1pt}\\
       #1 
    }%
    \hfill%
    \parbox{7cm}{
      \centering
      \rule{6cm}{1pt}\\
      #2
    }
  }
}


% Roman Numbers
\makeatletter
\newcommand{\rmnum}[1]{\romannumeral #1}
\newcommand{\Rmnum}[1]{\expandafter\@slowromancap\romannumeral #1@}
\makeatother

%protokollbeschreibungen
\newcommand{\protocolbrief}[4]{
\begin{center}
        \cornersize*{10pt}
        \Ovalbox{
          \parbox{9.95cm}{\hspace{-3mm}
            \begin{tabular}{l|l}
              \multicolumn{2}{p{10cm}}{\textsf{\large \bf {#1}}}\\\hline
              \textsf{\large{#2}}& \\\hline
              \multicolumn{2}{p{10cm}}{\textsf{\large{#3}}}\\\hline
              \multicolumn{2}{p{10cm}}{\textsf{\large{#4}}}
            \end{tabular}
          }
        }
\end{center}
\vspace{\baselineskip}
}

% method description
\newcommand{\method}[2]{
  \vspace{3mm}
  \begin{quotation}
    \noindent{\bf Method: #1}\vspace{2mm}
    
    #2
  \end{quotation}
}

% attack description
\newcommand{\attack}[2]{
  \vspace{3mm}
  \begin{quotation}
    \noindent{\bf Attack: #1}\vspace{2mm}
    
    #2
  \end{quotation}
}


% MATHEMATIC SYMBOLS
%% modular encapsulation of mathematical symbols to make
%% later changes as easy as possible 
%% (! Some parts of the document might be older than this functions !)
%%
%  \hashf   => \mathcal{H}
\newcommand{\hashf}{\ensuremath{\mH}}
%  \hash{r} => \hashf(r)
%\newcommand{\hash}[1]{\ensuremath{\hashf(#1)}}
%  \signf{n} => S^{-1}_{n}
\newcommand{\signf}[1]{\ensuremath{S^{-1}_{#1}}}
%  \sign{n}{m} => \signf{n}(m)
\newcommand{\sign}[2]{\ensuremath{\signf{#1}(#2)}}
%  \msign{n}{m} => \{m\}_{\signf{n}}
\newcommand{\msign}[2]{\ensuremath{\{#2\}_{\signf{#1}}}}
%  \cryptf{n} => K_{n}
\newcommand{\crypf}[1]{\ensuremath{K_{#1}}}
%  \dcryptf{n} =>
\newcommand{\dcrypf}[1]{\ensuremath{\crypf{#1}^{-1}}}
%  \crypt{n}{m} => \{m\}_{\crypf{n}}
\newcommand{\crypt}[2]{\ensuremath{\{#2\}_{\crypf{#1}}}}
%  \dcrypt{n}{m} => \{m\}_{\dcrypf{n}}
\newcommand{\dcrypt}[2]{\ensuremath{\{#2\}_{\dcrypf{#1}}}}


\newcommand{\eqbydef}[0]{\ensuremath{\mathop{=}\limits^{\textit{def.}}}}

%  \mP => \mathcal{P} (and others)
\newcommand{\mA}{\ensuremath{\mathcal{A}}}
\newcommand{\mB}{\ensuremath{\mathcal{B}}}
\newcommand{\mC}{\ensuremath{\mathcal{C}}}
\newcommand{\mD}{\ensuremath{\mathcal{D}}}
\newcommand{\mE}{\ensuremath{\mathcal{E}}}
\newcommand{\mF}{\ensuremath{\mathcal{F}}}
\newcommand{\mG}{\ensuremath{\mathcal{G}}}
\newcommand{\mH}{\ensuremath{\mathcal{H}}}
\newcommand{\mI}{\ensuremath{\mathcal{I}}}
\newcommand{\mK}{\ensuremath{\mathcal{K}}}
\newcommand{\mM}{\ensuremath{\mathcal{M}}}
\newcommand{\mN}{\ensuremath{\mathcal{N}}}
\newcommand{\mP}{\ensuremath{\mathcal{P}}}
\newcommand{\mQ}{\ensuremath{\mathcal{Q}}}
\newcommand{\mR}{\ensuremath{\mathcal{R}}}
\newcommand{\mS}{\ensuremath{\mathcal{S}}}
\newcommand{\mT}{\ensuremath{\mathcal{T}}}
\newcommand{\mU}{\ensuremath{\mathcal{U}}}
\newcommand{\mV}{\ensuremath{\mathcal{V}}}
\newcommand{\mX}{\ensuremath{\mathcal{X}}}
\newcommand{\mY}{\ensuremath{\mathcal{Y}}}


\newcommand{\E}{\ensuremath{\mathbb{E}}}
\newcommand{\F}{\ensuremath{\mathbb{F}}}
\newcommand{\N}{\ensuremath{\mathbb{N}}}
\newcommand{\Z}{\ensuremath{\mathbb{Z}}}
\newcommand{\Q}{\ensuremath{\mathbb{Q}}}
\newcommand{\R}{\ensuremath{\mathbb{R}}}

\newcommand{\gD}{\ensuremath{\mathfrak{D}}}
\newcommand{\gG}{\ensuremath{\mathfrak{G}}}
\newcommand{\gT}{\ensuremath{\mathfrak{T}}}

% other functions
\newcommand{\id}[1]{\ensuremath{\mathop{id}({#1})}}


%\newcommand{\qed}[0]{\hfill\ensuremath{\Box}}

%  \newcommand{\qed}{\nobreak \ifvmode \relax \else
%        \ifdim\lastskip<1.5em \hskip-\lastskip
%        \hskip1.5em plus0em minus0.5em \fi \nobreak
%        \vrule height0.75em width0.5em depth0.25em\fi}

%%% Local Variables: 
%%% mode: latex
%%% TeX-master: "main"
%%% End: 
